\documentclass[a4paper,11pt]{article} %scrartcl
%\setkomafont{disposition}{\normalfont\bfseries}
\usepackage[utf8]{inputenc}
\usepackage[backend=biber,bibencoding=utf8]{biblatex}
\usepackage{booktabs,array,dcolumn,graphicx,parskip,ifikompendiumforside,siunitx}
\bibliography{ref}

\title{How social networks influence open source software development}
%\subtitle{Subtitle}
\date{\today}
\author{John Rongved \and Lars Tveito \and Kristian A. Hiorth}

\begin{document}
%\maketitle
\ififorside{}

\tableofcontents{}

\section{Introduction}

The reason we chose this subject is that we perceive this to be an
interesting, relevant and important aspect of open-source
community. Historically, communication between participants in open-source
project has mainly been through mailing lists. During the previous decade, a
number of websites have been founded that provide developers with multiple
specialized channels of communications. Websites such as
Reddit\footnote{\texttt{http://www.reddit.com}},
StackOverflow\footnote{\texttt{http://www.stackoverflow.com}},
GitHub\footnote{\texttt{http://www.github.com}} and
HackerNews\footnote{\texttt{https://news.ycombinator.com}} have become
extremely popular since their creation and continues to grow. Each of these
websites have their strengths in user collaboration.


\subsubsection*{Reddit}
The site also known as the 'front page of the internet’ is a hugely popular
social website that provide its users with a platform for sharing content
and express opinions. The site does not itself host content, it relies on
other websites for media content sharing. Videos tend to be from YouTube,
images are from Imgur, sound tends to be on SoundCloud and code is almost
exclusively shared on GitHub. In order to organize content for their users
(~175M unique users per month) the page is divided into ‘subreddits’. The
subreddit for programming(r/programming) consists of about \num{500000}
subscribers, making it an attractive group to generate traffic for a
project. %% The way Reddit works is by a voting system that works in a way that
%% ‘good’ content becomes more visible, and ‘bad’ content is displaced.
Reddit works by a voting system that prioritizes 'good' content, and 'bad'
content is displaced. This way, if a website is prominently displayed, it is
usually the sign that this is a interesting website in many peoples opinion.


Reddit also functions as sort of a funnel for traffic, where the ‘good’
content quickly rises to the top of the page, resulting in more
traffic. This in turn spikes traffic on the site that hosts this content,
sometimes resulting in what is know as the ‘Reddit hug of death’, where a
small website hosting popular content crashes due to the spike in traffic
generated by Reddit. %% As such, Reddit functions as a tastemaker in new and
%% interesting content.


We will not do any investigation into non-technical subreddits, nor discuss
any of them. We do think technical subreddits have an effect on CBPP, but we
have no reason to believe that less technical subreddits (like
birdswitharms) have any impact on CBPP.

\subsubsection*{Hacker News}
The news aggregator Hacker News is a social news website that caters to
programmers and entrepreneurs. In a similar in structure to Reddit, Hacker
News uses ‘upvoting’ as a way to ensure quality content is displayed most
prominently on the site. The absence of a 'downvote', along with more
comprehensive guidelines\footnote{\texttt{https://news.ycombinator.com/newsguidelines.html}},
makes the community perhaps more stable, compared to Reddit.


\subsubsection*{StackOverflow}

The social Q\&A website for programmers, hobbyists, students or anyone
seeking technical advice. StackOverflow has quickly become the number one
place to seek information related to programming related problems. On the
website users submit a question and ‘tags’ it with relevant categories.
Similarly to Reddit posts are up- or down-voted by other users. If a
submitted question is too similar to an already answered thread, it may be
merged, or if the question is badly worded or too banal, it may be
deleted. If a question avoids these pitfalls, it may be answered by
anyone. Quality responses that many users find useful will be upvoted to the
top, and the user who submitted that answer will be awarded ‘rep points’ to
indicate his/hers level of expertise. This reward system is frequently used
in similar social websites, and is a form of ‘gamification’.
%this aspect will be explored more detailed in the final submission.
%% Dette må vi huske på!

\subsubsection*{GitHub}

Github is a Git repository web-based hosting service which offers its users
repositories for revision control and source control management. The site
has gained popularity by providing powerful tools, whilst staying easy to
use. It revolves around Git which is an free and open source distributed
source control management system, and plays a large role in teaching how to
use the system. Since it’s no less powerful than it’s predecessors, but is a
lot easier to use, we suspect it has lowered the bar to start making
contributions to open sourced projects. For all public repositories the
service is free, and in this way it pays a large contribution to open source
development. One can also keep a private repository on their server, but
this is in exchange for a fee. The site has by 2014 grown to be the largest
code host in the world, and has over 3,4 million users. In 2013 Github
announced that they had reached 10 million public repositories, which
accumulates to a very large contribution to the open source community.

\section{Facts and related works}

Traditionally, Free and Open Source Software (FOSS) projects have
relied upon mailing lists as the primary means of communication in
their communities. %CITEHERE%

Over the last several years, however, there has been a shift in the
way communication happens on the Internet, with the advent of the
Social Networks such as Facebook, Twitter and reddit. The latter, and
other special interest social media sites like Hacker News, have seen
particularly high adoption rates amongst software developers and power
users, also for relatively technical discussions.
% Kan jeg si det?

Additionally, several services which one would consider part of Social
Media have been launched that are specifically aimed at software
development, such as GitHub and StackOverflow, amongst many others.

% CITE \/
There has been a certain amount of research into FOSS projects and
their organizational behaviour, but much of this work is starting to
become dated and has very much been centered around mailing lists and
centralized (source) version control systems. Whilst mailing
lists undoubtedly remain a very important communication channel for
most projects, they are being challenged by social media and community
behaviour may be changing because of it.

While research into social media is quite hot, it's intersection with
software development, and indeed FOSS projects has not seen very much
published research. However, there have been a handful of interesting
works during the last years.

In a recent case study of the R community
\cite{Vasilescu14StackOverflow}, Vasilescu et al. demonstrate how that
project has seen a marked migration from traditional FOSS community
support in the form of a mailing list, to community support through
the Social Q\&{}A site StackOverflow. StackOverflow is built around
gamification which appears to actually have changed the
behaviour of community members.
%Cite gamification



The increasing popularity and migration to distributed version control
systems (dCVS) is also, in some cases, changing the way code actually
makes it way into projects. Examples of dCVS are Git, Mercurial
(coloquially known as Hg) and Bazaar. These have in common that they
are less reliant on a single point of coordination during active
development, which can make it much easier for small time contributors
to work on code changes, since they can work locally under source
control, or even set up third party repositories synchronized with the
upstream.

Github not only offers gratis\footnote{Github's services are free of
  charge for publically available repositories} hosting of Git dCVS
repositories, but also provides a number of other code collaboration
tools, like a graphical web interface to Git, issue tracking and code
review tools. This has the potential to ease the burden of starting
new projects, as well as contributing to existing ones that use
Github's platform - project instigators need not spend time on setting
up complicated infrastructure, and newcomers get access to good tools
off the bat.

It should be noted that while Github revolves around Git, which itself
is Free Software\footnote{Git is licensed under the GNU General Public
  License version 2.0}, the Github platform, including all the tools
and data generated by them, is actually proprietary. This poses a
dilemma for FOSS projects, since they may often not wish to become
reliant on proprietary software (and indeed, a whole proprietary third
party hosted platform).

\section{Discussion}

Knowing that social media is having an effect on FOSS projects in some
ways, we would have liked to be able to investigate whether this
effect stretched into the realm of development. In particular, it's
known in the community that receiving attention on certain of the most
popular software development social medias like (parts of) reddit and
Hacker News tends to lead to heavy traffic spikes for the exposed
project. Beyond this, there is little knowledge about longer term
effects. Do projects really gain more users, and do these spikes lead
to recruitment of new contributors?

In order to attempt to answer these questions, we contacted GitHub to
inquire about the possibility of receiving some general traffic
statistics, so we would be able to detect and document the occurence
of the aforementioned social media spikes. Unfortunately, this turned
out not to be possible because of privacy issues.
%Utbroder mer, foreslå ny approach, forklar ikke tid

\section{Concluding remarks}

\printbibliography
 
\end{document}
