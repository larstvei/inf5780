\documentclass[a4paper,11pt]{article} %scrartcl
%\setkomafont{disposition}{\normalfont\bfseries}
\usepackage[utf8]{inputenc}
\usepackage[backend=biber,bibencoding=utf8]{biblatex}
\usepackage{booktabs,array,dcolumn,graphicx,parskip,ifikompendiumforside,siunitx}
\usepackage[perpage,symbol*]{footmisc}
\usepackage{hyperref}
\bibliography{ref}

\title{Social Media influence on \hbox{Free and Open Source} Software Projects}
\subtitle{INF5780 Project 2 -- Autumn 2014}
\date{\today}
\author{John Rongved \and Lars Tveito \and Kristian A. Hiorth}

\begin{document}
%\maketitle
\ififorside{}

\pagenumbering{roman}
\tableofcontents{}
\newpage

\pagenumbering{arabic}
\section{Introduction}

Traditionally, Free and Open Source Software (FOSS) projects have
relied upon mailing lists as the primary means of communication in
their communities. %CITEHERE%

Over the last several years, however, there has been a shift in the way
communication happens on the Internet, with the advent of the Social
Networks such as Facebook\footnote{\texttt{http://www.facebook.com}}, Twitter\footnote{\texttt{http://www.twitter.com}} and reddit\footnote{\texttt{http://www.reddit.com}}.
The latter, and other special
interest social media sites like Hacker News\footnote{\texttt{https://news.ycombinator.com}}, appear to have particularly
high adoption rates amongst software developers and power users, also for
relatively technical discussions.
% Kan jeg si det?

Additionally, several services which one would consider part of Social Media
have been launched that are specifically aimed at software development, such
as GitHub\footnote{\texttt{http://www.github.com}} and Stack Overflow\footnote{\texttt{http://www.stackoverflow.com}}, amongst many others.
%Our focus in this essay will be on how social media influences different aspects of
%FOSS development.

%Open-source software development is often a highly collaborative activity,
%and effective communication between members and coordination of tasks is
%critical for a successful project.  Historically, mailing lists have been
%the preferred medium for coordinating development and user support
%activities\cite{Vasilescu14StackOverflow}. In recent years however,
%developers are moving away from mailing lists as the preferred means of
%communication.
%Social media has changed the landscape of the internet in several ways, and
%developers are taking advantage of new opportunities. In this report our
%focus will be on the role social media sites play in software
%development. Initially, we will describe some of the most common websites
%used for open-source development, also briefly present relevant litterature
%on the subject and finally discuss our findings.

\subsection{Research question}

FOSS development is based around a community of
contributors, and although management may be very different from commercial
ones, communication within the community is paramount. Social Media has
changed the landscape of the Internet in several ways, with an increased
focus on communities, and user generated content\cite{Kaplan201059}.
Based on the assumption that Social Media has reshaped the way Internet communities
operate and communicate, and that FOSS development is highly based on communal efforts
we have formulated the following research question:
%Social Media undoubtedly has had an effect on internet communities, with
%the key elements being content, communities and networks and Web 2.0 technologies\cite{Kaplan201059}.
%Several Social Media websites targeted specifically at the software development community is
%challenging traditional developer means of communications.
\begin{quote}
  \textit{How does Social Media influence communities involved in open source
    software development?}
\end{quote}

%This will be the basis of our inquiry into the topic, we will attempt to gain further 
%information on the subject

%\clearpage

\subsection{Important terms}

To clarify, we define some important terms that will be employed in this essay.

\begin{description}
\item[Social media] \hfill
  
  Andreas Kaplan and Michael Haenlein define social media as "a group of
  Internet-based applications that build on the ideological and
  technological foundations of Web 2.0, and that allow the creation and
  exchange of user-generated content."\cite{Kaplan201059} A key aspect is
  user generated content being made much more available by infrastructural
  and technological progress in the recent decade. Some prominent examples
  of social media applications are FaceBook, Twitter and Tumblr.

\item[Free and Open source software] \hfill
  %Insert well defined description of FOSS(fra pensum?)

  When we use the term FOSS, we mean software projects that have publicly
  available source code, open for modification and redistribution by
  anyone\cite[.p~18,p.~57]{leister2014opensource}. Free software and open
  source are not equivalent terms; free software is necessarily also open
  source, but open source does not have to follow the four essential
  freedoms\footnote{\texttt{https://www.gnu.org/philosophy/free-sw.html}}. The
  most general definition of FOSS is given by the Open Source Initiative
  (OSI)\footnote{\texttt{http://opensource.org/osd}}.

\end{description}

\subsection{Key aspects of open source development}

The focus in this essay will be on how social media influences open software
development. To assess this we have chosen some focus points that we feel
are representative in FOSS.

\begin{description}
\item[Exposure] \hfill
  
  A typical FOSS project does not have access to traditional means of
  promotion or advertisement. This fact is problematic because the community
  aspect of FOSS development is critical. Therefore a project needs a way to
  attract users as well as contributors to the project.
\item[Knowledge base] \hfill

  In any project, there is a need to share and keep track of
  knowledge. Documentation is not usually sufficient to cover this
  requirement by itself.
\item[Development tools] \hfill
  
  Practically all software projects require some form of collaboration
  tools, such as Version Control Systems (VCS) and Issue (Defect)
  Trackers. FOSS projects have a compounded need for good such tools, since
  the developers mostly are geographically distributed and cannot
  collaborate face to face.
\end{description}




\section{Central social media sites}

\subsection{Reddit}

Reddit is a social networking and news site that allow users to submit
content such as text or links to multimedia content. Reddit is organized
into categories, or 'subreddits', the subreddit for
programming(\texttt{r/programming}) consists of about \num{500000}
subscribers.
%The site does not itself host multimedia content,but users provide links
%with interesting content from sites like
%YouTube\footnote{\texttt{http://www.youtube.com}},
%Imgur\footnote{\texttt{http://www.imgur.com}} and
%SoundCloud\footnote{\texttt{http://www.soundcloud.com}}.
%% The way Reddit works is by a voting system that works in a way that
%% ‘good’ content becomes more visible, and ‘bad’ content is displaced.

Reddit employs a system in which registered users can cast votes on the
quality of content. Ranking of content is decided by a score based on the
amount of positive votes(upvotes) and negative votes(downvotes). Reddit has
a high number of users, around 175 million unique users every month. Because
of Reddit's large userbase it has the potential to generate huge traffic
spikes, in some cases even crashing smaller sites because of the overload of
traffic. This phenomenon is playfully referred to as the 'reddit hug of
death’.

%We will not do any investigation into non-technical subreddits, nor discuss
%any of them. We do think technical subreddits have an effect on CBPP, but we
%have no reason to believe that less technical subreddits (like
%birdswitharms) have any impact on CBPP.

\subsection{Hacker News}
The news aggregator Hacker News is a social news website that caters to
programmers and entrepreneurs. In a similar in structure to Reddit, Hacker
News uses ‘upvoting’ as a way to ensure quality content is displayed most
prominently on the site. The absence of a 'downvote', along with more
comprehensive
guidelines\footnote{\texttt{https://news.ycombinator.com/newsguidelines.html}},
makes the community less volatile in comparison the Reddit.


\subsection{Stack Overflow}

The social Q\&A website for programmers, hobbyists, students or anyone
seeking technical advice. Stack Overflow has quickly become the number one
place to seek information about programming related problems. On the website
users submit a question and ‘tags’ it with relevant categories.  Similarly
to Reddit posts are voted up or down by users. If a submitted question is
too similar to an already answered thread, it may be merged with a the
previous thread. If a submitted question is badly formatted, poorly worded
or disruptive, it may be deleted by a moderator.  However, if a submitter
avoids these pitfalls, it may be answered by anyone. Quality responses that
many users find useful will be upvoted to the top, and the user who
submitted that answer will be awarded ‘rep points’ to indicate his/hers
level of expertise.

%This reward system is frequently used in similar social websites,
%and is a form of ‘gamification’.
%this aspect will be explored more detailed in the final submission.
%% Dette må vi huske på!

\subsection{GitHub}

GitHub is a Git repository web-based hosting service which offers its users
repositories for revision control and source control management. The site
has gained popularity by providing powerful tools, whilst staying easy to
use. It revolves around Git which is an free and open source distributed
source control management system, and plays a large role in teaching how to
use the system. Since it’s no less powerful than it’s predecessors, but is a
lot easier to use, we suspect it has lowered the bar to start making
contributions to open sourced projects. For all public repositories the
service is free, and in this way it pays a large contribution to open source
development. One can also keep a private repository on their server, but
this is in exchange for a fee. The site has by 2014 grown to be the largest
code host in the world, and has over 3,4 million users. In 2013 GitHub
announced that they had reached 10 million public repositories, which
accumulates to a very large contribution to the open source community.

\section{Related works and opinions}
% CITE \/
There has been quite a bit of research into FOSS projects and their
organizational behaviour, but much of this work may be starting to show its
age and has very much been centered around mailing lists (such as
\cite{Oezbek10Cancer, singh2011network}) and centralized (source) version
control systems. Whilst mailing lists undoubtedly remain a very important
communication channel for most projects, they are being challenged by social
media and community behaviour may be changing because of it.

While research into social media is quite hot, its intersection with
software development, and indeed FOSS projects has not seen very much
published research. However, there have been a handful of interesting works
during the last years.

\subsection{Exposure -- Getting Attention}
FOSS projects, like all CBPP projects, are nothing without people
forming a community around it. It is from its community, generally
mostly users, that the project can recruit active contributors that
help keep it moving forward and staying alive. In order to foster a
community, projects need to be able to get word out about their
existence, and about why they are interesting.

There have been some sites dedicated to this, such as (the now
shuttered) FreshMeat, or SourceForge, in addition to the various
technical news venues. However, social media present new opportunities
in this domain. Programming/IT related subreddits, and especially
Hacker News (and other similar social media) gather a large number of
individuals with the interest and skills needed to participate in FOSS
projects, not to mention people responsible to fulfill computing
needs, either for themselves or for their employers.  On social media,
project members or anyone else are free to just post about interesting
projects, forgoing the need for editorial approval and
filtering. Usually, there is some form of crowdsourced moderation or
promotion of posts, so that content deemed interesting by many gains
more exposure. This might be an advantage for smaller projections,
especially, that do not have any brand name recognition to allure tech
journalists.

One especially interesting point is that, according to Singh et
al. \cite{singh2011network}, inter-project relationships between developers
may be quite beneficial to their success. If social media and coding
platforms can help foster such relationships, which they would seem suited
to do, this would be a big plus for the FOSS community.

\subsection{Knowledge base -- Community support}

In a recent case study of the R statistical programming language's community
\cite{Vasilescu14StackOverflow}, Vasilescu et al. demonstrate how that
project has seen a marked migration from traditional FOSS community support
in the form of a mailing list, to community support through the Social
Q\&{}A site Stack Overflow. Stack Overflow is built around the concept of
gamification\cite{deterding2011game} which appears to actually have changed
the behaviour of community members. Through correlating email addresses, the
authors showed that a significant number of informating providers (community
members who answer questions) participated in both the mailing list and on
Stack Overflow, and that the same individuals would tend to answer much
faster on the latter site. This can be explained by the fact that on
Stack Overflow, good answerers are rewarded with points, prominently
displayed to other community members. Thusly, they may feel more motivated
to provide timely responses to questions posed there, since the reward is
higher.

Stack Overflow has a very defined communication form, questions and
answers. Therefore it is very suited to typical support/knowledge sharing
activities. It is harder to imagine such benefits of gamification applied to
less didactic communication, such as technical deciision making, design
issues, etc. These discussions will tend to be conducted among peers, whilst
support usually is between knowledge seekers and providers, who are not
equal. Furthermore, there may be no ``right'' answer, so awarding points and
such would be very difficult. For these reasons, it is likely that sites
such as Stack Overflow will merely serve as a supplement to traditional
mailing lists (or similar fora).

Even though Stack Overflow is limited to a didactic questions and answers
style, it has been shown, in aggregate, to form rather good documentation of
for example Application Programming Interfaces (APIs)
\cite{parnin2012crowd}.

The data making up Stack Overflow also happens to be licensed under an open
license, Creative Commons Attribution-ShareAlike 3.0 Unported (BY-SA
3.0). Stack Overflow currently packages and archives all of it's data
quarterly, in collaboration with the Internet
Archive\footnote{\url{https://archive.org/details/stackexchange}}. This
ensures that communities are actually able extract and keep using this
amassed knowledge even outside of the Stack Overflow platform.


\subsection{Development tools -- Collaboration and Social Coding}

The increasing popularity and migration to distributed Version Control
Systems (dVCS) is also, in some cases, changing the way code actually makes
it way into projects. Examples of dVCS are Git, Mercurial (coloquially known
as Hg) and Bazaar. These have in common that they are less reliant on a
single point of coordination during active development, which can make it
much easier for small time contributors to work on code changes, since they
can work locally under source control, or even set up third party
repositories synchronized with the upstream.

GitHub not only offers gratis\footnote{GitHub's services are free of charge
  for publically available repositories} hosting of Git dVCS repositories,
but also provides a number of other code collaboration tools, like a
graphical web interface to Git, issue tracking and code review tools. This
has the potential to ease the burden of starting new projects, as well as
contributing to existing ones that use GitHub's platform - project
instigators need not spend time on setting up complicated infrastructure,
and newcomers get access to good tools off the bat.

It should be noted that while GitHub revolves around Git, which itself is
Free Software\footnote{Git is licensed under the GNU General Public License
  version 2.0}, the GitHub platform, including all the tools and data
generated by them, is actually proprietary. This poses a dilemma for FOSS
projects, since they may often not wish to become reliant on proprietary
software (and indeed, a whole proprietary third party hosted platform).


\section{Discussion}

\subsection{Traffic spikes on GitHub}

Knowing that social media is having an effect on FOSS projects in some ways,
we would have liked to be able to investigate whether this effect stretched
into the realm of development. In particular, it's known in the community
that receiving attention on certain of the most popular software development
social medias like (parts of) reddit and Hacker News tends to lead to heavy
traffic spikes for the exposed project. Beyond this, there is little
knowledge about longer term effects. Do projects really gain more users, and
do these spikes lead to recruitment of new contributors?

In order to attempt to answer these questions, we contacted GitHub to
inquire about the possibility of receiving some general traffic statistics,
so we would be able to detect and document the occurence of the
aforementioned social media spikes. Unfortunately, this turned out not to be
possible because of privacy issues.

It may still be possible to infer some data about this from publically
available data sources, such as GitHub
Archive\footnote{\url{http://www.githubarchive.org}}, which chronicles all
the publically visible events taking place on GitHub. If one correlates
these with archives of posts on reddit and Hacker News (which are also
available), one could compare activity in the time before and after a
high-ranked mention on these sites. We did not have time within the scope of
this project to pursue this any further however, since this approach would
likely have neccesitated a lot of work to build credible and informative
data.

Another interesting observation is made by Vasilescu et al. in
\cite{Vasilescu2013Overlap}: they were able to show an overlap in the
userbases of Stack Overflow and GitHub of almost a quarter of the
users, by comparing email addresses. Since users might not sign up
using the same addresses on both sites, this is likely an
underestimate. This is definite evidence of a strong relation between
these two services.

\subsection{The Emacs community}

We wanted to look at a few existing projects and see how Social Media has
influenced the community and project development. We were already to some
extent a part of the Emacs community, and we thought this would make it
easier to reach out to it.

Emacs, originally developed by Richard Stallman in 1976, currently has
\num{191} active project
members\footnote{\texttt{http://savannah.gnu.org/project/memberlist.php?group=emacs}}.
Since \num{2012} Emacs has featured it's own package manager, making
external packages more easily available. This has extended Emacs' ecosystem
beyond the development of the Emacs core, and people are increasingly
depending on external packages.

The most popular repository for external packages is
Melpa\footnote{\texttt{http://melpa.org} - at 11. November 2014} and
currently contains \num{2076} external packages; of which \num{1762} are
mirrored from GitHub and \num{220} are mirrored form
EmacsWiki\footnote{\texttt{http://www.emacswiki.org/}}. The total number of
libraries in EmacsWiki is \num{1952}, which is relatively few compared to
the swarming number of GitHub repository's, which is
\num{21152}\footnote{\href{https://github.com/search?utf8=\%E2\%9C\%93\&q=language\%3Aemacs-lisp\&type=Repositories\&ref=searchresults}{https://github.com/search} - at 11. November 2014}.

The presence of GitHub seems to have made a serious impact to the number of
openly available projects for this particular community. We want to argue
that this development also has an indirect effect on the development of the
core of Emacs.

Richard Stallman has spoke of the development of external packages several
times on the Emacs mailing list. His general opinion is that libraries that
other packages depend on should be included in the Emacs
core\footnote{\texttt{https://lists.gnu.org/archive/html/emacs-devel/2014-09/msg00630.html}},
but also shows skepticism toward this
development\footnote{\texttt{https://lists.gnu.org/archive/html/emacs-devel/2014-09/msg00582.html}}.

One consequence of external packages is that it makes changing the
interfaces in Emacs more difficult, because the maintainers has no control
over these packages. Because the copyright is held by whomever submitted the
package, the packages can't simply be included in Emacs. He also expresses
some concern that these external packages speaks against the GNU projects
values, such as recommending use of non-free software.

\subsection{Gamification}

These sites all include some kind of gamification elements. Blah blah.

\subsection{Who will benefit?}

We've seen that Social Media provides new community dynamics, and that
these could likely be quite beneficial to FOSS projects. However, it
is far from certain that this applies to all projects.

As is well known, FOSS projects are vastly different, from one man
shows with minuscule user bases through to gargantuan projects such as
Linux, backed by a number of multinational corporations and deployed
in all manners of critical infrastructure.

To us it appears natural that Social Media offer larger benefits to
projects on the smaller end of the scale, since the large and well
known projects already have access to vast infrastructure and are in
reality established brands in their own right. This ought to be true
especially when projects seek to attract exposure, since traditional
technology journalists would probably prefer covering projects that
many of their readers already know.

Furthermore, managing complex collaboration tools such as VCS and code
review systems can be very time consuming. Large projects often have
contributors who spend all their time dedicated to the projects taking
care of such administrative tasks, a luxury few emerging projects can
afford. This means fully featured open platforms such as Stack
Overflow and GitHub can free up significant developer time for doing
actual development on the software itself.

\subsection{Risks}

The saying goes that there is no such thing as a free lunch, and one
might do well to keep that in mind when considering Social
Media. While services are often offered free of charge or very
modestly priced, the service providers obviously need to generate
income somehow. Frequently, this will be accomplished by monetizing
user data.
%Burde egentlig ha referanse der, men finner ikke :(

Not only does one have to accept that one's data might be sold, but
there is the very real added risk of platform lock-in. If a project
becomes so reliant on a particular tool, such as GitHub, it may prove
hard to migrate off it in case conditions change or a better
alternative surfaces. This is less of a concern for pure information
channels like reddit and Hacker News, since historical data will be
far less important.

\section{Concluding remarks}

\newpage
\printbibliography

\end{document}
