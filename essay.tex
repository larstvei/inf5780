\documentclass[a4paper,11pt,oneside]{book} %scrartcl
%\setkomafont{disposition}{\normalfont\bfseries}
\usepackage[utf8]{inputenc}
\usepackage[backend=biber,bibencoding=utf8,sorting=none]{biblatex}
\usepackage{graphicx}
\usepackage{booktabs,array,dcolumn}
\usepackage{ifikompendiumforside}

 
\title{How sossialnetworks influence open source software development}
\subtitle{Subtitle}
\author{John Rongved, Lars Tveito, Kristian A. Hiorth}
\date{\today}
\pagestyle{plain}

\begin{document}
%\maketitle
\ififorside

\frontmatter

  \tableofcontents

  \mainmatter

\chapter*{Introduction}

The reason we chose this subject is that we perceive this to be an interesting, relevant and important aspect of open-source community. Historically, communication between participants in open-source project has mainly been through e-mail lists. During the previous decade, a number of websites have been founded that provide developers with multiple specialized channels of communications. Websites such as  Reddit\footnote{\texttt{http://www.reddit.com}}, StackOverflow\footnote{\texttt{http://www.stackoverflow.com}}, GitHub\footnote{\texttt{http://www.github.com}} and HackerNews\footnote{\texttt{https://news.ycombinator.com}} have become extremely popular since their creation and continues to grow. Each of these websites have their strengths in user collaboration. 
\\\\
\textbf{Reddit}, also know as the ‘front page of the internet’ is a hugely popular social website that provide its users with a platform for sharing content and express opinions. The site does not itself host content, it relies on other websites for media content sharing. Videos tend to be from YouTube, images are from Imgur and sound tends to be on SoundCloud and code is almost exclusively shared on GitHub. In order to organize content for their users(~175M unique users per month) the page is divided into ‘subreddits’. The subreddit for programming(r/programming) consists of about 500,000 subscribers, making it an attractive group to generate traffic for a project. The way Reddit works is by a voting system that works in a way that ‘good’ content becomes more visible, and ‘bad’ content is displaced. This way, if a website is prominently displayed, it is usually the sign that this is a interesting website in many peoples opinion.
\\
Reddit also functions as sort of a funnel for traffic, where the ‘good’ content quickly rises to the top of the page, resulting in more traffic. This in turn spikes traffic on the site that hosts this content, sometimes resulting in what is know as the ‘Reddit hug of death’, where a small website hosting popular content crashes due to the spike in traffic generated by Reddit. As such, Reddit functions as a tastemaker in new and interesting content. 
\\\\
\textbf{Hacker News} is a social news website that caters to programmers and entrepreneurs. In a similar in structure to Reddit, Hacker News uses ‘upvoting’ as a way to ensure quality content is displayed most prominently on the site.
\\\\
\textbf{StackOverflow} is a social Q\&A website for programmers, hobbyists, students or anyone seeking technical advice. StackOverflow has quickly become the number one place to seek information related to programming related problems. On the website users submit a question and ‘tags’ it with relevant categories and similarly to Reddit it is up or down voted by other users. If a submitted question is too similar to an already answered thread, it may be merged. Or if the question is badly worded or too banal, it may be deleted. If a question avoids these pitfalls, it may be answered by anyone. Quality responses that many users find useful will be upvoted to the top, and the user who submitted that answer will be awarded ‘rep points’ to indicate his/hers level of expertise. This reward system is frequently used in similar social websites, and is a form of ‘gamification’, this aspect will be explored more detailed in the final submission.
\\\\
\textbf{GitHub} is a Git repository web-based hosting service which offers its users repositories for revision control and source control management. The site has gained popularity by providing powerful tools, whilst staying easy to use. It revolves around Git which is an free and open source distributed source control management system, and plays a large role in teaching how to use the system. Since it’s no less powerful than it’s predecessors, but is a lot easier to use, we suspect it has lowered the bar to start making contributions to open sourced projects. For all public repositories the service is free, and in this way it pays a large contribution to open source development. One can also keep a private repository on their server, but this is in exchange for a fee. The site has by 2014 grown to be the largest code host in the world, and has over 3,4 million users. In 2013 Github announced that they had reached 10 million public repositories, which accumulates to a very large contribution to the open source community.








\newpage{}

\chapter*{Social media facts}

\chapter*{Discussion}

\chapter*{Concluding remarks}

\backmatter

\printbibliography
 
\end{document}
